\addcontentsline{toc}{chapter}{はじめに}
\chapter*{はじめに}

本論執筆の最大の動機は, 私が行った研究内容すべての引き継ぎである. 私は卒業してしまうので, 今後研究に着手する後輩に向けて細かな応答ができない. 本論を読めば, 誰でも研究内容を再現できるようになることを目指している. 

1章にはこれから扱う系についての説明を記述している. この系をLAMMPS上で再現するには, 掲載しているLAMMPSファイルを実際に動かしてみればいい. 

2章には実際に行った各シミュレーションの詳細な設定と, 重心位置の推移のみを掲載した. なお付録にはすべてのシミュレーションから得られたデータを用いて分析した結果の画像を添付している. 

3章では2章で得たデータを踏まえて, 本論の主張の部分を記述している. 

4章は本論のまとめである. 2ページにまとめているが, 詳細を知りたい場合は, 1章から3章を参照するようにしてほしい. 

この卒業論文を作成するにあたって, 使用した\TeX ファイルなど, 研究発表に用いたスライドのデータ, 研究手法などをまとめたMarkdownファイルは以下のURLから, または中川研が管理しているSSDからダウンロードができるようになっている. (ただし実験データは中川研管理のSSDにしか保存していない.)

\href{https://github.com/m-agnet/Report.git}{https://github.com/m-agnet/Report.git}

\href{https://github.com/m-agnet/Presen.git}{https://github.com/m-agnet/Presen.git}

\href{https://github.com/m-agnet/HowTo.git}{https://github.com/m-agnet/HowTo.git}