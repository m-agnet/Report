\chapter{数値実験の結果}\label{chap:simulation}

この章では, 行ったシミュレーションの設定についてそれぞれ説明をして, その結果得た各系の重心位置の推移を示す.

以下のように, $\text{R}_\text{a}$と$\text{R}_\text{t}$を少しずつ変えた系を設定して, 25種類の系でそれぞれシミュレーションをした. これ以降, 簡単のために各系のことをそれぞれの図のキャプションに示されたアルファベットを用いて示すことがある. 例えば, 25個の系を用意した場合は表 \ref{tb:system} のように呼ぶことがある. 

\vspace{1\baselineskip}

\begin{table}
  \centering
  \begin{tabular}{|c|c|c|c|c|c|} \hline
          & $\text{R}_\text{a}:0.0$ & $\text{R}_\text{a}:0.4693$ & $\text{R}_\text{a}:0.9387$ & $\text{R}_\text{a}:1.408$ & $\text{R}_\text{a}:1.877$ \\ \hline
    $\text{R}_\text{t}:0.0$ & a      & b      & c      & d      & e     \\ \hline
    $\text{R}_\text{t}:0.125$ & f      & g      & h      & i      & j     \\ \hline
    $\text{R}_\text{t}:0.25$ & k      & l      & m      & n      & o     \\ \hline
    $\text{R}_\text{t}:0.375$ & p      & q      & r      & s      & t     \\ \hline
    $\text{R}_\text{t}:0.5$ & u      & v      & w      & x      & y     \\ \hline
  \end{tabular} 
  \caption{}
  \label{tb:system}
\end{table}

\section{重力と熱流を同時にかける}\label{sec:RaRtmap}

\vspace{1\baselineskip}

まずは, 重力と熱流を同じタイミングでかけたシミュレーションを用意した. 粒子数は1250個にして, その他のパラメータについては先行研究\cite{Yoshida}を参考にした. 

\begin{itemize}
  \item $N = 1250$
  \item $\rho {\sigma}^2 = 0.4$
  \item $L_x / \sigma \simeq 39.5$
  \item $L_y / \sigma \simeq 79.0$
  \item $k_{\text{B}} T / \varepsilon = 0.43$
  \item $k_{\text{B}} \Delta T / \varepsilon = 0.04$
  \item $mg\sigma/\varepsilon \simeq 4.0 \times 10^{-4}$
  \item $_i \sqrt{\varepsilon / m \sigma^2}= 4.0 \times 10^4$
  \item $t_f \sqrt{\varepsilon / m \sigma^2} = 2.0 \times 10^{5}$
\end{itemize}

図\ref{fig:RaRtmap_movie}で定常状態($t\sqrt{\varepsilon / m \sigma^2} = 2.0 \times 10^{5}$)時点でのスナップショットを見る.

\input{subtex/RaRtmap_movie.tex}

スナップショットだけではダイナミクスがわからないので, 次に図\ref{fig:RaRtmap_time}で重心位置$Y_g$(式\eqref{CoM})の時間変化の様子を見る. 

\begin{align}
  \label{CoM}
  Y_g &\equiv \bar{y_i} = \frac{1}{N} \sum_{i}^{N} y_i
\end{align}

重心位置$Y_g$を, 系の$y$幅を用いて$0\sim 1$にスケーリングして, 時系列プロットしている.

\input{subtex/RaRtmap_time.tex}

図\ref{fig:RaRtmap_time}では, 初期条件から$t_f \sqrt{\varepsilon / m \sigma^2} = 2.0 \times 10^{5}$までの時間変化をプロットしており, 赤い線の時刻($t \sqrt{\varepsilon / m \sigma^2}= 4.0 \times 10^4$)あたりで定常状態に到達していると判断した. 図\ref{fig:RaRtmap_time}を見ると, 引力幅$\text{R}_\text{a}=0.0$の時は, ある定常状態のまわりのランダムなゆらぎが観測されているが, $\text{R}_\text{a} \ge 0.469, \text{R}_\text{t} \ge 0.125$の時は決してランダムとは言えない. $\text{R}_\text{a} \ge 0.938 かつ\text{R}_\text{t} \ge 0.250$の時には周期的なダイナミクスが発生していることが見てとれる. 

\section{重力を先にかけて, 熱流を後からかける}\label{sec:RaRtmap_drop}

初期条件が異なる場合にダイナミクスが変化するかを調べるために, 重力と熱流をかけるタイミングをずらしたシミュレーションを設定した. まず, $t \sqrt{\varepsilon / m \sigma^2} = 2.0 \times 10^{5}$の時点まで重力のみをかけて, 粒子集団が落ちきってから熱流をかける. 

\begin{itemize}
  \item $N = 1250$
  \item $\rho \sigma^2 = 0.4$
  \item $L_x / \sigma \simeq 79.0$
  \item $L_y / \sigma \simeq 158.1$
  \item $k_{\text{B}} T/\varepsilon = 4.3$
  \item $k_{\text{B}} \Delta T/\varepsilon = 0.0$
  \item $mg\sigma/\varepsilon \simeq 2.0 \times 10^{-4}$
\end{itemize}

温度差のある熱浴をそれぞれ改めて以下のようにつけ, 熱流をかけてシミュレーションを続ける. 熱流をかけ始めてから, 定常状態に至るまでの時間を\ref{sec:RaRtmap}節と同様に$t \sqrt{\varepsilon / m \sigma^2} = 4.0 \times 10^{4}$として, 重力をかけ始めてから数えて$t \sqrt{\varepsilon / m \sigma^2} = 2.4 \times 10^{5}$の時点で定常状態に到達していると考える. 

\begin{itemize}
  \item $\chi = k_{\text{B}}\Delta T / mg L_y = 1.265$
  \item $k_{\text{B}} \Delta T/\varepsilon = 0.04$
  \item $t_i \sqrt{\varepsilon / m \sigma^2} = 2.4 \times 10^{5}$
  \item $t_f \sqrt{\varepsilon / m \sigma^2} = 4.0 \times 10^{5}$
\end{itemize}

\input{subtex/RaRtmap_drop_time.tex}

\ref{sec:RaRtmap}節と同様に, 図\ref{fig:RaRtmap_drop_time}を見たとき, 引力幅$\text{R}_\text{a}=0.0$の時は, ある定常状態のまわりのランダムなゆらぎが観測されているが, $\text{R}_\text{a} \ge 0.469, \text{R}_\text{t} \ge 0.125$の時は決してランダムとは言えない. $\text{R}_\text{a} \ge 0.938 かつ\text{R}_\text{t} \ge 0.250$の時には周期的なダイナミクスが発生していることが見てとれる.  

このことより, 重力と熱流をかけるタイミングをずらしたとしてもダイナミクスは大きく変わらないことが分かる. 

図\ref{fig:RaRtmap_time}, \ref{fig:RaRtmap_drop_time}の (x), (y) に注目すると, \ref{sec:RaRtmap}節でのそれと比べて, 上壁に張りついている時間が長い様子が見てとれる. 周期的なダイナミクスが発生しているのかどうか, \ref{sec:RaRtmap10}節では, \ref{sec:RaRtmap10}の10倍の時間をとって実験をする. 

\section{重力のみをかける}\label{sec:dT0}

重力のみをかけた系, 熱流のみをかけた系を考えると, それぞれ流体系は壁の下部, 上部で安定することが予想できたが, 実際に実験で確かめた. 

まず, 重力のみをかけた系であるが, $t \sqrt{\varepsilon / m \sigma^2}= 4.0 \times 10^4$の時点で定常状態に到達していると考える. 

\begin{itemize}
  \item $N = 1250$
  \item $\rho {\sigma}^2 = 0.4$
  \item $L_x / \sigma \simeq 39.5$
  \item $L_y / \sigma \simeq 79.0$
  \item $k_{\text{B}} T / \varepsilon = 0.43$
  \item $k_{\text{B}} \Delta T / \varepsilon = 0.0$
  \item $mg\sigma/\varepsilon \simeq 4.0 \times 10^{-4}$
  \item $t_i \sqrt{\varepsilon / m \sigma^2}= 4.0 \times 10^4$
  \item $t_f \sqrt{\varepsilon / m \sigma^2} = 2.0 \times 10^{5}$
\end{itemize}


\input{subtex/dT0_time.tex}

この重心位置の推移から, 重心のみをかけた系では流体系が壁の下部で安定することが確認できた. 

\section{熱流のみをかける}\label{sec:g0}

次に, 熱流のみをかけた系で実験をしたが, これも $t \sqrt{\varepsilon / m \sigma^2}= 4.0 \times 10^4$の時点で定常状態に到達していると考える. 

\begin{itemize}
  \item $N = 1250$
  \item $\rho {\sigma}^2 = 0.4$
  \item $L_x / \sigma \simeq 39.5$
  \item $L_y / \sigma \simeq 79.0$
  \item $k_{\text{B}} T / \varepsilon = 0.43$
  \item $k_{\text{B}} \Delta T / \varepsilon = 0.04$
  \item $mg\sigma/\varepsilon = 0.0$
  \item $t_i \sqrt{\varepsilon / m \sigma^2}= 4.0 \times 10^4$
  \item $t_f \sqrt{\varepsilon / m \sigma^2} = 2.0 \times 10^{5}$
\end{itemize}

\input{subtex/g0_time.tex}

これを見ると, $\text{R}_\text{a} \ge 0.938 かつ \text{R}_\text{t} \ge 0.250$の時には流体系が壁の上部で安定することが確認できた.

$\text{R}_\text{a}=0.0$の系では壁にWCAポテンシャルが設定されるので, 流体系は上下の壁から斥力のみがはたらく. このとき重心位置は図\ref{fig:g0_time}の(a),(f),(k),(p),(u)のように上下にゆらぐことが分かる.

\section{重力と熱流を同時にかける(時間10倍)}\label{sec:RaRtmap10}

\ref{sec:RaRtmap_drop}節でも述べたように, \ref{sec:RaRtmap}節での系それぞれについてのシミュレーションの時間を10倍に増やして, 周期的なダイナミクスが生じているかを見る. 

\begin{itemize}
  \item $N = 1250$
  \item $\rho {\sigma}^2 = 0.4$
  \item $L_x / \sigma \simeq 39.5$
  \item $L_y / \sigma \simeq 79.0$
  \item $k_{\text{B}} T / \varepsilon = 0.43$
  \item $k_{\text{B}} \Delta T / \varepsilon = 0.04$
  \item $mg\sigma/\varepsilon \simeq 4.0 \times 10^{-4}$
  \item $t_i \sqrt{\varepsilon / m \sigma^2}= 2.4 \times 10^5$
  \item $t_f \sqrt{\varepsilon / m \sigma^2} = 2.0 \times 10^{6}$
\end{itemize}

\input{subtex/RaRtmap10_time.tex}

$\text{R}_\text{a} \ge 0.938 かつ\text{R}_\text{t} \ge 0.250$の時には周期的なダイナミクスが発生していることが見てとれる. \ref{sec:RaRtmap}節よりも(x),(y)の系について周期的なダイナミクスを確認することができた. 

\section{重力を先にかけて, 熱流を後からかける(時間10倍)}\label{sec:RaRtmap10_drop}

\ref{sec:RaRtmap_drop}節での(q),(r),(s)の系に注目すると, 周期的なダイナミクスが(r),(s)は見てとれるが, (q)の系はそうではないので, これら3つの系について, シミュレーションをとる時間を10倍にした実験をした. 

先ほどと同様に重力のみをかけて, 粒子集団が落ちきってから熱流をかけると同時に測定を開始する.

\begin{itemize}
  \item $N = 1250$
  \item $\rho \sigma^2 = 0.4$
  \item $L_x / \sigma \simeq 39.5$
  \item $L_y / \sigma \simeq 79.0$
  \item $k_{\text{B}} T/\varepsilon = 4.3$
  \item $k_{\text{B}} \Delta T/\varepsilon = 0.0$
  \item $mg\sigma/\varepsilon \simeq 2.0 \times 10^{-4}$
\end{itemize}

続いて重力をかけた緩和後の系で, 温度差のある熱浴をそれぞれ改めて以下のようにつけ, 熱流をかけてシミュレーションをする. 

\begin{itemize}
  \item $\chi = k_{\text{B}}\Delta T / mg L_y = 1.265$
  \item $k_{\text{B}} \Delta T/\varepsilon = 0.04$
  \item $t_i \sqrt{\varepsilon / m \sigma^2} = 2.4 \times 10^{5}$
  \item $t_f \sqrt{\varepsilon / m \sigma^2} = 2.2 \times 10^{6}$
\end{itemize}

\ref{sec:RaRtmap_drop} 節での(q),(r),(s)と本節の(a),(b),(c)を比べて(順に壁の濡れ性が対応している.)見てみると, (b),(c)は周期的なダイナミクスがはっきりと現れているのに対して, (a)の系では定常状態のまわりのランダムなゆらぎが観測されている. これまでのことからも, 濡れ性を大きくすると周期的なダイナミクスが現れると予想できる. それでは, 壁の濡れ性を変えることによる流体系のダイナミクスの特徴づけをこれから行いたい. 

\begin{figure}[H]
  \centering
  \includegraphics[scale=0.6]{image/qrs10_drop_time/2023-12-28T10:59:29.242_qrs_gap_chi1.265_Ay50_rho0.4_T0.43_dT0.04_Rd0.0_Rt0.375_Ra0.4693845_g0.0003999718779659611_run4.0e8.png}
  \label{}
  \caption{$\text{R}_\text{a}=0.469,\text{R}_\text{t}=0.375$}
\end{figure}

\begin{figure}[H]
  \centering
  \begin{tabular}{ccc}
    \begin{minipage}[t]{0.2\hsize}
      \centering
      \includegraphics[width=\textwidth]{image/qrs10_drop_time/2023-12-28T10:59:29.242_qrs_gap_chi1.265_Ay50_rho0.4_T0.43_dT0.04_Rd0.0_Rt0.375_Ra0.4693845_g0.0003999718779659611_run4.0e8.png}
      \subcaption{$\text{R}_\text{a}=0.469,\\\text{R}_\text{t}=0.375$}
      \label{}
    \end{minipage} &
    \begin{minipage}[t]{0.2\hsize}
      \centering
      \includegraphics[width=\textwidth]{image/qrs10_drop_time/2023-12-28T10:59:29.748_qrs_gap_chi1.265_Ay50_rho0.4_T0.43_dT0.04_Rd0.0_Rt0.375_Ra0.938769_g0.0003999718779659611_run4.0e8.png}
      \subcaption{$\text{R}_\text{a}=0.938,\\\text{R}_\text{t}=0.375$}
      \label{}
    \end{minipage} &
    \begin{minipage}[t]{0.2\hsize}
      \centering
      \includegraphics[width=\textwidth]{image/qrs10_drop_time/2023-12-28T10:59:29.846_qrs_gap_chi1.265_Ay50_rho0.4_T0.43_dT0.04_Rd0.0_Rt0.375_Ra1.4081535_g0.0003999718779659611_run4.0e8.png}
      \subcaption{$\text{R}_\text{a}=1.408,\\\text{R}_\text{t}=0.375$}
      \label{}
    \end{minipage} 
  \end{tabular}
  \caption{$t_i = 2.4 \times 10^5 , t_f = 2.2 \times 10^6, t\sqrt{\epsilon/m{\sigma}^2} = 2000$ごとにプロット.}
  \label{fig:qrs10_drop_time}
\end{figure}