\chapter{数値実験の結果}\label{sec:simulation}

この章では, 行ったシミュレーションの設定についてそれぞれ説明をして, その結果得た各系の重心位置の推移を示す.

% \section{先行研究}

% 壁を完全に濡らしている状態を考えたいので, $r^{\text{wall}}_{\text{cut}}=3.0\sigma$ に設定する.

% また, 重力をかけた状態で粒子が下に落ちきり, 定常状態にあるとみなせるまでシミュレーションを行ってから, 熱流をかけている.\cite{Yoshida}

% \begin{itemize}
%   \item $N = 5000$
%   \item $\rho \sigma^2 = 0.4$
%   \item $L_x / \sigma \simeq 79.0$
%   \item $L_y / \sigma \simeq 158.1$
%   \item $k_{\text{B}} T/\varepsilon = 4.3$
%   \item $k_{\text{B}} \Delta T/\varepsilon = 0.0$
%   \item $mg\sigma/\varepsilon \simeq 2.0 \times 10^{-4}$
%   \item ${t_f}^{\prime} \sqrt{\varepsilon / m \sigma^2} = 5.0 \times 10^{5}$
% \end{itemize}

% \begin{figure}[H]
%   \centering
%   \includegraphics[scale=0.2]{image/drop5000.png}
%   \caption{$t \sqrt{\varepsilon / m \sigma^2} = 5.0 \times 10^{5}$でのスナップショット}
%   \label{}
% \end{figure}

% 続いて重力をかけた緩和後の系で, 温度差のある熱浴をそれぞれ改めて以下のようにつけ, 熱流をかけてシミュレーションをしている. 

% \begin{itemize}
%   \item $\chi = k_{\text{B}}\Delta T / mg L_y = 1.265$
%   \item $k_{\text{B}} \Delta T/\varepsilon = 0.04$
%   \item $t_i \sqrt{\varepsilon / m \sigma^2} = 5.0 \times 10^{5}$
%   \item $t_f \sqrt{\varepsilon / m \sigma^2} = 1.0 \times 10^{6}$
% \end{itemize}

% $(\text{R}_\text{t} = 0.5, \text{R}_\text{a} = 3.0 - 2^{1/6})$ で実験を行う. このときには, $(\varepsilon^{\text{wall}} = \varepsilon, \sigma^{\text{wall}} = \sigma, r^{\text{wall}}_{\text{cut}} = 3.0 \sigma^{\text{wall}})$ になるので, ${\phi_{\text{LJ}}^{\text{wall}}} = \phi_{\text{LJ}}$ ということになり, 先行研究と同じシミュレーションを行うことができる.

% \begin{figure}[H]
%   \centering
%   \href{https://youtu.be/CIEyUPvPY6A}{\includegraphics[scale=0.2]{image/2023-11-21T21:01:17.543_followup_chi1.265_Ay100_rho0.4_T0.43_dT0.04_Rd0.0_Rt0.5_Ra1.877538_g0.00019998593898298055_run1.0e8_output.png}}
%   \caption{$t \sqrt{\varepsilon / m \sigma^2} = 1.0 \times 10^{6}$でのスナップショット}
%   \label{}
% \end{figure}

以下のように, $\text{R}_\text{a}$と$\text{R}_\text{t}$を少しずつ変えた系を設定して, 25種類の系でそれぞれシミュレーションをした.

\vspace{1\baselineskip}

\begin{center}
\begin{tabular}{|c|c|c|c|c|c|} \hline
        & $\text{R}_\text{a}:0.0$ & $\text{R}_\text{a}:0.4693$ & $\text{R}_\text{a}:0.9387$ & $\text{R}_\text{a}:1.408$ & $\text{R}_\text{a}:1.877$ \\ \hline
  $\text{R}_\text{t}:0.0$ & a      & b      & c      & d      & e     \\ \hline
  $\text{R}_\text{t}:0.125$ & f      & g      & h      & i      & j     \\ \hline
  $\text{R}_\text{t}:0.25$ & k      & l      & m      & n      & o     \\ \hline
  $\text{R}_\text{t}:0.375$ & p      & q      & r      & s      & t     \\ \hline
  $\text{R}_\text{t}:0.5$ & u      & v      & w      & x      & y     \\ \hline
\end{tabular}
\end{center}

\section{重力と熱流を同時にかける}\label{sec:RaRtmap}

\vspace{1\baselineskip}

まずは, 重力と熱流を同じタイミングでかけたシミュレーションを用意した. 粒子数については1250個と少なめではあるが, その他のパラメータについては先行研究\cite{Yoshida}を参考にした. 

\begin{itemize}
  \item $N = 1250$
  \item $\rho {\sigma}^2 = 0.4$
  \item $L_x / \sigma \simeq 39.5$
  \item $L_y / \sigma \simeq 79.0$
  \item $k_{\text{B}} T / \varepsilon = 0.43$
  \item $k_{\text{B}} \Delta T / \varepsilon = 0.04$
  \item $mg\sigma/\varepsilon \simeq 4.0 \times 10^{-4}$
  \item $t_f \sqrt{\varepsilon / m \sigma^2} = 2.0 \times 10^{5}$
\end{itemize}

図\ref{fig:RaRtmap_movie}で定常状態($t\sqrt{\varepsilon / m \sigma^2} = 2.0 \times 10^{5}$)時点でのスナップショットを見る.

\input{subtex/RaRtmap_movie.tex}

スナップショットだけではダイナミクスがわからないので, 次に図\ref{fig:RaRtmap_time}で重心位置$Y_g$(式\eqref{CoM})を使って, 時間変化の様子を見る. 

\begin{align}
  \label{CoM}
  Y_g &\equiv \bar{y_i} = \frac{1}{N} \sum_{i}^{N} y_i
\end{align}

重心位置$Y_g$を, 系の$y$幅を用いて$0\sim 1$にスケーリングして, 時系列プロットしている.

\input{subtex/RaRtmap_time.tex}

図\ref{fig:RaRtmap_time}では, 初期条件から$t_f \sqrt{\varepsilon / m \sigma^2} = 2.0 \times 10^{5}$までの時間変化をプロットしており, 赤い線の時刻あたりで定常状態に到達していると判断した. 

図\ref{fig:RaRtmap_time}を見ると, 引力幅$\text{R}_\text{a}=0.0$の時は, ある定常状態のまわりのランダムなゆらぎが観測されているが, $\text{R}_\text{a} \ge 0.938, \text{R}_\text{t} \ge 0.25$の時は決してランダムとは言えない. 周期的なダイナミクスが発生していることが見てとれる.

\section{重力を先にかけて, 熱流を後からかける}

初期条件が異なる場合にダイナミクスが変化するかを調べたい. 

まず重力のみをかけて, 粒子集団が落ちきってから熱流をかけると同時に測定を開始する.

\begin{itemize}
  \item $N = 1250$
  \item $\rho \sigma^2 = 0.4$
  \item $L_x / \sigma \simeq 79.0$
  \item $L_y / \sigma \simeq 158.1$
  \item $k_{\text{B}} T/\varepsilon = 4.3$
  \item $k_{\text{B}} \Delta T/\varepsilon = 0.0$
  \item $mg\sigma/\varepsilon \simeq 2.0 \times 10^{-4}$
  \item ${t_f}^{\prime} \sqrt{\varepsilon / m \sigma^2} = 2.0 \times 10^{5}$
\end{itemize}

続いて重力をかけた緩和後の系で, 温度差のある熱浴をそれぞれ改めて以下のようにつけ, 熱流をかけてシミュレーションをする. 

\begin{itemize}
  \item $\chi = k_{\text{B}}\Delta T / mg L_y = 1.265$
  \item $k_{\text{B}} \Delta T/\varepsilon = 0.04$
  \item $t_i \sqrt{\varepsilon / m \sigma^2} = 2.4 \times 10^{5}$
  \item $t_f \sqrt{\varepsilon / m \sigma^2} = 4.0 \times 10^{5}$
\end{itemize}

\input{subtex/RaRtmap_drop_time.tex}



\section{重力のみをかける}

\begin{itemize}
  \item $N = 1250$
  \item $\rho {\sigma}^2 = 0.4$
  \item $L_x / \sigma \simeq 39.5$
  \item $L_y / \sigma \simeq 79.0$
  \item $k_{\text{B}} T / \varepsilon = 0.43$
  \item $k_{\text{B}} \Delta T / \varepsilon = 0.0$
  \item $mg\sigma/\varepsilon \simeq 4.0 \times 10^{-4}$
  \item $t_f \sqrt{\varepsilon / m \sigma^2} = 2.0 \times 10^{5}$
\end{itemize}

\input{subtex/dT0_time.tex}

\section{熱流のみをかける}

\begin{itemize}
  \item $N = 1250$
  \item $\rho {\sigma}^2 = 0.4$
  \item $L_x / \sigma \simeq 39.5$
  \item $L_y / \sigma \simeq 79.0$
  \item $k_{\text{B}} T / \varepsilon = 0.43$
  \item $k_{\text{B}} \Delta T / \varepsilon = 0.04$
  \item $mg\sigma/\varepsilon = 0.0$
  \item $t_f \sqrt{\varepsilon / m \sigma^2} = 2.0 \times 10^{5}$
\end{itemize}

\input{subtex/g0_time.tex}

\section{重力と熱流を同時にかける(時間10倍)}\label{sec:RaRtmap10}

\begin{itemize}
  \item $N = 1250$
  \item $\rho {\sigma}^2 = 0.4$
  \item $L_x / \sigma \simeq 39.5$
  \item $L_y / \sigma \simeq 79.0$
  \item $k_{\text{B}} T / \varepsilon = 0.43$
  \item $k_{\text{B}} \Delta T / \varepsilon = 0.04$
  \item $mg\sigma/\varepsilon \simeq 4.0 \times 10^{-4}$
  \item $t_f \sqrt{\varepsilon / m \sigma^2} = 2.0 \times 10^{6}$
\end{itemize}

\input{subtex/RaRtmap10_time.tex}

\section{重力を先にかけて, 熱流を後からかける(時間10倍)}

先ほどと同様に重力のみをかけて, 粒子集団が落ちきってから熱流をかけると同時に測定を開始する.

\begin{itemize}
  \item $N = 1250$
  \item $\rho \sigma^2 = 0.4$
  \item $L_x / \sigma \simeq 39.5$
  \item $L_y / \sigma \simeq 79.0$
  \item $k_{\text{B}} T/\varepsilon = 4.3$
  \item $k_{\text{B}} \Delta T/\varepsilon = 0.0$
  \item $mg\sigma/\varepsilon \simeq 2.0 \times 10^{-4}$
  \item ${t_f}^{\prime} \sqrt{\varepsilon / m \sigma^2} = 2.0 \times 10^{5}$
\end{itemize}

続いて重力をかけた緩和後の系で, 温度差のある熱浴をそれぞれ改めて以下のようにつけ, 熱流をかけてシミュレーションをする. 

\begin{itemize}
  \item $\chi = k_{\text{B}}\Delta T / mg L_y = 1.265$
  \item $k_{\text{B}} \Delta T/\varepsilon = 0.04$
  \item $t_i \sqrt{\varepsilon / m \sigma^2} = 2.4 \times 10^{5}$
  \item $t_f \sqrt{\varepsilon / m \sigma^2} = 2.2 \times 10^{6}$
\end{itemize}

\begin{figure}[H]
  \centering
  \includegraphics[scale=0.6]{image/qrs10_drop_time/2023-12-28T10:59:29.242_qrs_gap_chi1.265_Ay50_rho0.4_T0.43_dT0.04_Rd0.0_Rt0.375_Ra0.4693845_g0.0003999718779659611_run4.0e8.png}
  \label{}
  \caption{$\text{R}_\text{a}=0.469,\text{R}_\text{t}=0.375$}
\end{figure}

\begin{figure}[H]
  \centering
  \begin{tabular}{ccc}
    \begin{minipage}[t]{0.2\hsize}
      \centering
      \includegraphics[width=\textwidth]{image/qrs10_drop_time/2023-12-28T10:59:29.242_qrs_gap_chi1.265_Ay50_rho0.4_T0.43_dT0.04_Rd0.0_Rt0.375_Ra0.4693845_g0.0003999718779659611_run4.0e8.png}
      \subcaption{$\text{R}_\text{a}=0.469,\\\text{R}_\text{t}=0.375$}
      \label{}
    \end{minipage} &
    \begin{minipage}[t]{0.2\hsize}
      \centering
      \includegraphics[width=\textwidth]{image/qrs10_drop_time/2023-12-28T10:59:29.748_qrs_gap_chi1.265_Ay50_rho0.4_T0.43_dT0.04_Rd0.0_Rt0.375_Ra0.938769_g0.0003999718779659611_run4.0e8.png}
      \subcaption{$\text{R}_\text{a}=0.938,\\\text{R}_\text{t}=0.375$}
      \label{}
    \end{minipage} &
    \begin{minipage}[t]{0.2\hsize}
      \centering
      \includegraphics[width=\textwidth]{image/qrs10_drop_time/2023-12-28T10:59:29.846_qrs_gap_chi1.265_Ay50_rho0.4_T0.43_dT0.04_Rd0.0_Rt0.375_Ra1.4081535_g0.0003999718779659611_run4.0e8.png}
      \subcaption{$\text{R}_\text{a}=1.408,\\\text{R}_\text{t}=0.375$}
      \label{}
    \end{minipage} 
  \end{tabular}
  \caption{$t_i = 2.4 \times 10^5 , t_f = 2.2 \times 10^6, t\sqrt{\epsilon/m{\sigma}^2} = 2000$ごとにプロット.}
  \label{fig:qrs10_drop_time}
\end{figure}