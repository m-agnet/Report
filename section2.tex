\section{実験}\label{sec:simulation}

この章では, 行ったシミュレーションの設定についてそれぞれ説明をしていく.

系の上下両端のポテンシャルエネルギー差$mgL_y$と運動エネルギー差$k_{\text{B}}\Delta T$の比を$\chi$として以下のように設定する.

\begin{align}
  \chi &\equiv \frac{k_{\text{B}}\Delta T}{mgL_{y}}
\end{align}

壁ポテンシャルまわりの無次元パラメータを3つ用意する.

\begin{align}
  \text{R}_\text{d} &: 乾き具合. \\
  \text{R}_\text{t} &: 壁の厚み. \\
  \text{R}_\text{a} &: 濡れ具合.
\end{align}

これらを用いて, 壁-粒子間相互作用LJポテンシャルは以下のように書き表す.

\begin{align}
  \varepsilon^{\text{wall}} &= \qty(1.0 - \text{R}_\text{d}) \times \varepsilon \\
  \sigma^{\text{wall}} &= \qty(0.5 + \text{R}_\text{t}) \times \sigma \\
  r^{\text{wall}}_{\text{cut}} &= \qty(2^{1/6} + \text{R}_\text{a}) \times \sigma^{\text{wall}}
\end{align}

パラメータ $(\text{R}_\text{d}, \text{R}_\text{t}, \text{R}_\text{a})$ を変えることによって, 壁-粒子間相互作用LJポテンシャルが変わる. このときに, どのようにして粒子集団の様相が変化するかをみる.

$\text{R}_\text{a}$ と $\text{R}_\text{t}$ を少しずつ変えた系でシミュレーションをして, 粒子集団の様相の変化を見たい. 以下に示すのが, それらを動かす範囲である.

\begin{align}
  \text{R}_\text{a} &\colon 0.0 \sim 3.0 - 2^{1/6} = 1.877538\dots \\
  \text{R}_\text{t} &\colon 0.0 \sim 0.5
\end{align}

\begin{figure}[H]
  \centering
  \begin{tabular}{ccc}
    \begin{minipage}[t]{0.2\hsize}
      \centering
      \includegraphics[width=\textwidth]{image/RaRtmap_LJ/LJ-Potential_Rt0.0.png}
      \subcaption{$\text{R}_\text{t}:0.0$}
      \label{}
    \end{minipage} &
    \begin{minipage}[t]{0.2\hsize}
      \centering
      \includegraphics[width=\textwidth]{image/RaRtmap_LJ/LJ-Potential_Rt0.125.png}
      \subcaption{$\text{R}_\text{t}:0.125$}
      \label{}
    \end{minipage} &
    \begin{minipage}[t]{0.2\hsize}
      \centering
      \includegraphics[width=\textwidth]{image/RaRtmap_LJ/LJ-Potential_Rt0.25.png}
      \subcaption{$\text{R}_\text{t}:0.25$}
      \label{}
    \end{minipage} \\
    \begin{minipage}[t]{0.2\hsize}
      \centering
      \includegraphics[width=\textwidth]{image/RaRtmap_LJ/LJ-Potential_Rt0.375.png}
      \subcaption{$\text{R}_\text{t}:0.375$}
      \label{}
    \end{minipage} &
    \begin{minipage}[t]{0.2\hsize}
      \centering
      \includegraphics[width=\textwidth]{image/RaRtmap_LJ/LJ-Potential_Rt0.5.png}
      \subcaption{$\text{R}_\text{t}:0.5$}
      \label{}
    \end{minipage} 
  \end{tabular}
  \caption{LJ-potential}
  \label{}
\end{figure}

本章の以降の実験は特記がない限り以下のパラメータに近い値で行うものとする. 

\begin{itemize}
  \item $N = 1250$: 粒子数
  \item $\rho {\sigma}^2 = 0.4$: 粒子数密度
  \item $L_x / \sigma = 39.528471 \simeq 39.5$: 系の$x$幅
  \item $L_y / \sigma = 79.0569414 \simeq 79.0$: 系の$y$幅
  \item $k_{\text{B}} T / \varepsilon = 0.43$: 初期温度
  \item $k_{\text{B}} \Delta T / \varepsilon = 0.04$: 熱浴の温度差
  \item $mg\sigma/\varepsilon = 0.0003999718779659611 \simeq 4.0 \times 10^{-4}$: 粒子にかかる重力の大きさ
  \item $\dd t \sqrt{\epsilon/m{\sigma}^2} = 0.005$: シミュレーションにおける時間刻み.
\end{itemize}


以下に記すのは, 今後解析をする際に示すシミュレーションについての時間に関する説明である.

\begin{align}
  t_i &\colon シミュレーション開始時から, 物理量を解析する際にデータを採用し始める時間. これ以降は定常状態であるとみなす. \\
  t_f & \colon シミュレーション開始時から, シミュレーションの終了時までの時間.
\end{align}

いずれの実験の場合も$t\sqrt{\epsilon/m{\sigma}^2}=0$の時点では粒子は以下の画像のように, 系に規則正しく並べられているとする.

\begin{figure}[H]
  \centering
  \includegraphics[scale=0.2]{image/initial1250.png}
  \caption{$N=1250$}
  \label{}
\end{figure}

\subsection{追実験}

壁を完全に濡らしている状態を考えたいので, $r^{\text{wall}}_{\text{cut}}=3.0\sigma$ に設定する.

また, 重力をかけた状態で粒子が下に落ちきり, 緩和しているとみなせるまでシミュレーションを行ってから, 熱流をかけている. この時点でのシミュレーションではデータを解析することはない.

\begin{itemize}
  \item $N = 5000$
  \item $\rho \sigma^2 = 0.4$
  \item $L_x / \sigma = 79.0\dots$
  \item $L_y / \sigma = 158.1\dots$
  \item $k_{\text{B}} T/\varepsilon = 4.3$
  \item $k_{\text{B}} \Delta T/\varepsilon = 0.0$
  \item $mg\sigma/\varepsilon = 2.0 \times 10^{-4}$
  \item $t_f \sqrt{\varepsilon / m \sigma^2} = 5.0 \times 10^{5}$
\end{itemize}

\begin{figure}[H]
  \centering
  \includegraphics[scale=0.2]{image/drop5000.png}
  \caption{$t \sqrt{\varepsilon / m \sigma^2} = 5.0 \times 10^{5}$でのスナップショット}
  \label{}
\end{figure}

続いて重力をかけた緩和後の系で, 熱浴の温度差を改めて以下のようにつけ, 熱流をかけてシミュレーションをしている. このシミュレーションでデータを解析する.

\begin{itemize}
  \item $\chi = k_{\text{B}}\Delta T / mg L_y = 1.265$
  \item $k_{\text{B}} \Delta T/\varepsilon = 0.04$
  \item $t_i \sqrt{\varepsilon / m \sigma^2} = 5.0 \times 10^{5}$
  \item $t_f \sqrt{\varepsilon / m \sigma^2} = 1.0 \times 10^{6}$
\end{itemize}

$(\text{R}_\text{d} = 0.0, \text{R}_\text{t} = 0.5, \text{R}_\text{a} = 3.0 - 2^{1/6})$ で実験を行う. これは, $(\varepsilon^{\text{wall}} = \varepsilon, \sigma^{\text{wall}} = \sigma, r^{\text{wall}}_{\text{cut}} = 3.0 \sigma^{\text{wall}})$ に対応しているので, ${\phi_{\text{LJ}}}^{\text{wall}} = \phi_{\text{LJ}}$ ということになり, 先行研究と同じ結果を得ることができる.

\begin{figure}[H]
  \centering
  \href{https://youtu.be/CIEyUPvPY6A}{\includegraphics[scale=0.2]{image/2023-11-21T21:01:17.543_followup_chi1.265_Ay100_rho0.4_T0.43_dT0.04_Rd0.0_Rt0.5_Ra1.877538_g0.00019998593898298055_run1.0e8_output.png}}
  \caption{Ay100\_rho0.4\_T0.43\_dT0.04\_Rd0.0\_Rt0.5\_Ra1.877538\_g0.0004\_run1.0e8}
  \label{}
\end{figure}

\subsection{重力と熱流を同時にかける}

以下のように, $\text{R}_\text{a}$と$\text{R}_\text{t}$を少しずつ変えた系を設定して, それぞれシミュレーションをした.

\vspace{1\baselineskip}

\begin{tabular}{|c|c|c|c|c|c|} \hline
        & $\text{R}_\text{a}:0.0$ & $\text{R}_\text{a}:0.469$ & $\text{R}_\text{a}:0.938$ & $\text{R}_\text{a}:1.408$ & $\text{R}_\text{a}:1.877$ \\ \hline
  $\text{R}_\text{t}:0.0$ & a      & b      & c      & d      & e     \\ \hline
  $\text{R}_\text{t}:0.125$ & f      & g      & h      & i      & j     \\ \hline
  $\text{R}_\text{t}:0.25$ & k      & l      & m      & n      & o     \\ \hline
  $\text{R}_\text{t}:0.375$ & p      & q      & r      & s      & t     \\ \hline
  $\text{R}_\text{t}:0.5$ & u      & v      & w      & x      & y     \\ \hline
\end{tabular}

\vspace{1\baselineskip}

パラメータを確認する.

\begin{itemize}
  \item $N = 1250$
  \item $\rho {\sigma}^2 = 0.4$
  \item $L_x / \sigma = 39.5\dots$
  \item $L_y / \sigma = 79.0\dots$
  \item $k_{\text{B}} T / \varepsilon = 0.43$
  \item $k_{\text{B}} \Delta T / \varepsilon = 0.04$
  \item $mg\sigma/\varepsilon = 4.0 \times 10^{-4}$
  \item $t_f \sqrt{\varepsilon / m \sigma^2} = 2.0 \times 10^{5}$
\end{itemize}

この際の粒子集団の様相は以下のようになる.

\input{subtex/RaRtmap_movie.tex}

重心位置$Y_g$を系の$y$幅でスケーリングして, 時系列プロットすると,

\begin{align}
  Y_g &\equiv \bar{y_i} = \frac{1}{N} \sum_{i}^{N} y_i
\end{align}

\input{subtex/RaRtmap_time.tex}


\subsection{重力を先にかけて, 熱流を後からかける}

以下は, 追実験のときと同じように, まず重力のみをかけて, 粒子集団が落ちきってから熱流をかけると同時に測定を開始するものである.

\input{subtex/RaRtmap_drop_time.tex}



\subsection{重力のみをかける}

\begin{itemize}
  \item $N = 1250$
  \item $\rho {\sigma}^2 = 0.4$
  \item $L_x / \sigma = 39.5\dots$
  \item $L_y / \sigma = 79.0\dots$
  \item $k_{\text{B}} T / \varepsilon = 0.43$
  \item $k_{\text{B}} \Delta T / \varepsilon = 0.0$
  \item $mg\sigma/\varepsilon = 4.0 \times 10^{-4}$
  \item $t_f \sqrt{\varepsilon / m \sigma^2} = 2.0 \times 10^{5}$
\end{itemize}

\subsection{熱流のみをかける}

\begin{itemize}
  \item $N = 1250$
  \item $\rho {\sigma}^2 = 0.4$
  \item $L_x / \sigma = 39.5\dots$
  \item $L_y / \sigma = 79.0\dots$
  \item $k_{\text{B}} T / \varepsilon = 0.43$
  \item $k_{\text{B}} \Delta T / \varepsilon = 0.04$
  \item $mg\sigma/\varepsilon = 0.0$
  \item $t_f \sqrt{\varepsilon / m \sigma^2} = 2.0 \times 10^{5}$
\end{itemize}


