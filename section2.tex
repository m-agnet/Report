\section{実験}

雨が降るシミュレーションをしたい.

系の両端のポテンシャルエネルギー差$mgL_y$と運動エネルギー差$k_{\text{B}}\Delta T$の比を$\chi$として以下のように設定する.

\begin{align}
  \chi \equiv \frac{k_{\text{B}}\Delta T}{mgL_{y}} = 1.265
\end{align}

壁ポテンシャルまわりの無次元パラメータを3つ用意する.

\begin{align}
  \text{R}_\text{d} &: 乾き具合. \\
  \text{R}_\text{t} &: 壁の厚み. \\
  \text{R}_\text{a} &: 濡れ具合.
\end{align}

これを用いて, 壁-粒子間相互作用LJポテンシャルは以下のように書き表す.

\begin{align}
  \varepsilon^{\text{wall}} &= \qty(1.0 - \text{R}_\text{d}) \times \varepsilon \\
  \sigma^{\text{wall}} &= \qty(0.5 + \text{R}_\text{t}) \times \sigma \\
  r^{\text{wall}}_{\text{cut}} &= \qty(2^{1/6} + \text{R}_\text{a}) \times \sigma^{\text{wall}}
\end{align}

パラメータ $(\text{R}_\text{d}, \text{R}_\text{t}, \text{R}_\text{a})$ を変えることによって, 壁-粒子間相互作用LJポテンシャルが変わったときに, どのように粒子集団の様相が変化するかをみる. 本章の以降の実験は特記がない限り以下のパラメータに近い値で行うものとする. 

\begin{itemize}
  \item $N = 1250$
  \item $\rho {\sigma}^2 = 0.4$
  \item $L_x / \sigma = 39.5\dots$
  \item $L_y / \sigma = 79.0\dots$
  \item $k_{\text{B}} T / \varepsilon = 0.43$
  \item $k_{\text{B}} \Delta T / \varepsilon = 0.04$
  \item $mg\sigma/\varepsilon = 4.0 \times 10^{-4}$
  \item $t_f \sqrt{\varepsilon / m \sigma^2} = 1.0 \times 10^{5}$
\end{itemize}


\subsection{追実験}

壁を完全に濡らしていることを考えたいので, $r^{\text{wall}}_{\text{cut}}=1.0$ のときを考えている.

重力をかけた状態で緩和するまでシミュレーションを行っている.

\begin{itemize}
  \item $N = 5000$
  \item $\rho \sigma^2 = 0.4$
  \item $L_x / \sigma = 79.0\dots$
  \item $L_y / \sigma = 158.1\dots$
  \item $k_{\text{B}} T/\varepsilon = 4.3$
  \item $k_{\text{B}} \Delta T/\varepsilon = 0.0$
  \item $mg\sigma/\varepsilon = 2.0 \times 10^{-4}$
  \item $t_b \sqrt{\varepsilon / m \sigma^2} = 5.0 \times 10^{5}$
\end{itemize}

重力をかけた緩和後の系で, 熱浴の温度差を改めて以下のようにつけ, 熱流を流してシミュレーションをしている.

\begin{itemize}
  \item $\chi = k_{\text{B}}\Delta T / mg L_y = 1.265$
  \item $k_{\text{B}} \Delta T/\varepsilon = 0.04$
  \item $t_a \sqrt{\varepsilon / m \sigma^2} = 5.0 \times 10^{5}$
\end{itemize}

まずは, $(\text{R}_\text{d} = 0.0, \text{R}_\text{t} = 0.5, \text{R}_\text{a} = 3.0 - 2^{1/6})$ で実験を行う. これは, $(\varepsilon^{\text{wall}} = \varepsilon, \sigma^{\text{wall}} = \sigma, r^{\text{wall}}_{\text{cut}} = 3\sigma^{\text{wall}})$ に対応しているので, $\phi_{\text{LJ}} = \phi_{\text{LJ}}^{\text{wall}}$ ということになり, 先行研究と同じ結果を得るはずである.

\begin{figure}[H]
  \centering
  \href{https://youtu.be/CIEyUPvPY6A}{\includegraphics[scale=0.2]{image/2023-11-21T21:01:17.543_followup_chi1.265_Ay100_rho0.4_T0.43_dT0.04_Rd0.0_Rt0.5_Ra1.877538_g0.00019998593898298055_run1.0e8_output.png}}
  \caption{Ay100\_rho0.4\_T0.43\_dT0.04\_Rd0.0\_Rt0.5\_Ra1.877538\_g0.0004\_run1.0e8}
  \label{}
\end{figure}


\subsection{$\text{R}_\text{a}$, $\text{R}_\text{t}$ マップ}

\begin{align}
  \text{R}_\text{a} &= 0.0 \sim 3.0 - 2^{1/6} \\
  \text{R}_\text{t} &= 0.0 \sim 0.5
\end{align}

\begin{figure}[H]
  \centering
  \begin{tabular}{ccc}
    \begin{minipage}[t]{0.2\hsize}
      \centering
      \includegraphics[width=\textwidth]{image/RaRtmap_LJ/LJ-Potential_Rt0.0.png}
      \subcaption{$\text{R}_\text{t}:0.0$}
      \label{}
    \end{minipage} &
    \begin{minipage}[t]{0.2\hsize}
      \centering
      \includegraphics[width=\textwidth]{image/RaRtmap_LJ/LJ-Potential_Rt0.125.png}
      \subcaption{$\text{R}_\text{t}:0.125$}
      \label{}
    \end{minipage} &
    \begin{minipage}[t]{0.2\hsize}
      \centering
      \includegraphics[width=\textwidth]{image/RaRtmap_LJ/LJ-Potential_Rt0.25.png}
      \subcaption{$\text{R}_\text{t}:0.25$}
      \label{}
    \end{minipage} \\
    \begin{minipage}[t]{0.2\hsize}
      \centering
      \includegraphics[width=\textwidth]{image/RaRtmap_LJ/LJ-Potential_Rt0.375.png}
      \subcaption{$\text{R}_\text{t}:0.375$}
      \label{}
    \end{minipage} &
    \begin{minipage}[t]{0.2\hsize}
      \centering
      \includegraphics[width=\textwidth]{image/RaRtmap_LJ/LJ-Potential_Rt0.5.png}
      \subcaption{$\text{R}_\text{t}:0.5$}
      \label{}
    \end{minipage} 
  \end{tabular}
  \caption{LJ-potential}
  \label{}
\end{figure}

\begin{itemize}
  \item $N = 1250$
  \item $\rho {\sigma}^2 = 0.4$
  \item $L_x / \sigma = 39.5\dots$
  \item $L_y / \sigma = 79.0\dots$
  \item $k_{\text{B}} T / \varepsilon = 0.43$
  \item $k_{\text{B}} \Delta T / \varepsilon = 0.04$
  \item $mg\sigma/\varepsilon = 4.0 \times 10^{-4}$
  \item $t_f \sqrt{\varepsilon / m \sigma^2} = 2.0 \times 10^{5}$
\end{itemize}

\input{subtex/RaRtmap_movie}

重心位置をスケーリングして, 時系列プロットすると,

\input{subtex/RaRtmap_time.tex}

重心位置の揺らぎについて同時プロットで表してみる. 図\ref{fig:RaRtmap_time_Ra0.0_Rt0.0}から$\tau \sqrt{\varepsilon / m \sigma^2} = 2.5 \times 10^{4}$からのデータを用いてプロットすることにしている.

\begin{figure}[H]
  \begin{tabular}{cc}
    \begin{minipage}[t]{0.5\hsize}
      \centering
      \includegraphics[width=\textwidth]{image/lnStdYg_Ra_ti25000.png}
      \subcaption{}
      \label{}
    \end{minipage}
    \begin{minipage}[t]{0.5\hsize}
      \centering
      \includegraphics[width=\textwidth]{image/lnStdYg_Rt_ti25000.png}
      \subcaption{}
      \label{}
    \end{minipage}
  \end{tabular}
  \caption{}
  \label{}
\end{figure}

\subsection{周期性}

重心位置が周期的に変化しているのかを調べたい.

図\ref{fig:RaRtmap_time_Ra0.469_Rt0.500}と図\ref{fig:RaRtmap_time_Ra0.938_Rt0.500}の間をもっと詳しく見る.

\begin{figure}[H]
  \centering
  \begin{tabular}{ccccc}
    \begin{minipage}[t]{0.2\hsize}
      \centering
      \includegraphics[width=\textwidth]{image/RaRtmap_time/2023-11-22T18:19:26.970_RaRt_chi1.265_Ay50_rho0.4_T0.43_dT0.04_Rd0.0_Rt0.5_Ra0.4693845_g0.0003999718779659611_run1.2e8.png}
      \subcaption{Ra0.4693845}
      \label{}
    \end{minipage} &
    \begin{minipage}[t]{0.2\hsize}
      \centering
      \includegraphics[width=\textwidth]{image/RaRtmap_time/2023-11-22T18:19:27.571_RaRt_chi1.265_Ay50_rho0.4_T0.43_dT0.04_Rd0.0_Rt0.5_Ra0.586730625_g0.0003999718779659611_run1.2e8.png}
      \subcaption{Ra0.586730625}
      \label{}
    \end{minipage} &
    \begin{minipage}[t]{0.2\hsize}
      \centering
      \includegraphics[width=\textwidth]{image/RaRtmap_time/2023-11-22T18:19:27.653_RaRt_chi1.265_Ay50_rho0.4_T0.43_dT0.04_Rd0.0_Rt0.5_Ra0.70407675_g0.0003999718779659611_run1.2e8.png}
      \subcaption{Ra0.70407675}
      \label{}
    \end{minipage} &
    \begin{minipage}[t]{0.2\hsize}
      \centering
      \includegraphics[width=\textwidth]{image/RaRtmap_time/2023-11-22T18:19:27.720_RaRt_chi1.265_Ay50_rho0.4_T0.43_dT0.04_Rd0.0_Rt0.5_Ra0.821422875_g0.0003999718779659611_run1.2e8.png}
      \subcaption{Ra0.821422875}
      \label{}
    \end{minipage} &
    \begin{minipage}[t]{0.2\hsize}
      \centering
      \includegraphics[width=\textwidth]{image/RaRtmap_time/2023-11-22T18:19:27.787_RaRt_chi1.265_Ay50_rho0.4_T0.43_dT0.04_Rd0.0_Rt0.5_Ra0.938769_g0.0003999718779659611_run1.2e8.png}
      \subcaption{Ra0.938769}
      \label{}
    \end{minipage} 
  \end{tabular}
  \caption{}
  \label{}
\end{figure}
