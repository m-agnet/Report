\documentclass[dvipdfmx]{jsarticle}

% 枠
\usepackage{fancybox}
\usepackage{ascmac}
% 色
\usepackage{color}
% 数式
\usepackage{amsmath}
\usepackage{amsfonts}
\usepackage{mathtools}
\usepackage{bm,physics}
\usepackage{siunitx}
\usepackage[thicklines]{cancel}
% 画像
\usepackage[dvipdfmx]{graphicx} 
% \usepackage[draft]{graphicx} % 画像出力を枠だけにする.
\usepackage[hang,small,bf]{caption}
\usepackage[subrefformat=parens]{subcaption}
\usepackage{here}
% グラフ
\usepackage{tikz}
\usetikzlibrary{
  intersections,
  calc,
  arrows.meta
}
% ソースコード
\usepackage{listings,jlisting}
\lstset{
  language=C++,
	stringstyle={\ttfamily},
	commentstyle={\ttfamily},
	basicstyle={\ttfamily},
	columns=fixed,
  frame={tb},
  breaklines=true,
  columns=[l]{fullflexible},
	backgroundcolor=\color[gray]{.90}, % pdfをコピペしたときに行番号を巻き込まないようにする.
  numbers=left, % 行数を表示したければonにする.
  xrightmargin=0em,
  xleftmargin=3em,
  numberstyle={\scriptsize},
  stepnumber=1,
  numbersep=1em,
	tabsize=2,
  lineskip=-0.5ex
}
% アンカー
\usepackage[dvipdfmx]{hyperref}
\usepackage{pxjahyper}
\hypersetup{
  setpagesize=false,
  bookmarksnumbered=true,
  bookmarksopen=true,
  colorlinks=true,
  linkcolor=black,
  citecolor=red,
  urlcolor=magenta
}
% 数式相互参照
\usepackage{cleveref}
\usepackage{autonum}
\numberwithin{equation}{subsection}
% 目次に参考文献を入れる. これをonにすると目次の体裁が崩れてしまう.
% \usepackage{tocbibind}

\begin{document}

% 表紙
\title{重力と熱流下における粒子集団の様相}
\author{理学部理学科物理学コース 学籍番号20S2035Y 山本 凜}
\date{\today}
\maketitle
\newpage
% 目次
\setcounter{tocdepth}{3}
\tableofcontents
\newpage
% 本文

\section{系の設定}

2次元の気液共存系で, 質量$m$の粒子が$N$個存在することを考え, 系の上下には壁, 左右には周期境界条件を課す. また, 重力を$y$軸正の向きにかけて, 熱流を$y$軸負の向きに流す. この熱流は, 系の上下の領域にそれぞれ異なる温度を設定したlangevin熱浴を使用することによってかけることとし, NVT-MDシミュレーションを実行する. また, 各熱浴の$y$幅は$8\sigma$となるように設定する. (図\ref{fig:system})


\begin{figure}[H]
  \centering
  \caption{}
  \label{fig:system}
  \includegraphics[scale=0.7]{image/system.jpg}
\end{figure}

\subsection{ハミルトニアン}

結論は, 

\begin{align}
  H(\Gamma; g)
  &= \sum_{i=1}^{N}
  \left[
    \frac{{\bm{p}_i}^2}{2m} 
    + \sum_{j > i}^{N}
      \tilde{\phi}_{\text{LJ}}^{\text{pair}}(r_{ij})
    + mgy_i
    + V^{\text{wall}} (y_i)
  \right] . \ \tag*{\eqref{Hamiltonian}} 
\end{align}

以降, 本節はこれに至るまでの過程を述べる.

\subsubsection{粒子-粒子間相互作用ポテンシャル}

シミュレーションを行う際に, 典型的な粒子間相互作用ポテンシャルとして, 12-6 Lennard-Jones Potential を採用する.

\begin{align}
  \phi_{\text{LJ}}^{\text{pair}}(r; \varepsilon, \sigma) = 4\varepsilon \qty[\qty(\frac{\sigma}{r})^{12} - \qty(\frac{\sigma}{r})^{6} ] \\
\end{align}

シミュレーション上では, カットオフ長 $r_{\text{cut}}^{\text{pair}}=3\sigma$ とポテンシャルのシフトアップを考慮して

\begin{align}
  \tilde{\phi}_{\text{LJ}}^{\text{pair}}(r;r_{\text{cut}}^{\text{pair}}) = \qty{\phi_{\text{LJ}}^{\text{pair}}(r) - \phi_{\text{LJ}}^{\text{pair}}(r_{\text{cut}}^{\text{pair}})}\theta \qty(r_{\text{cut}}^{\text{pair}}-r) \\
\end{align}

のように書き換えたポテンシャルを用いている.

\subsubsection{周期境界条件と最近接イメージ規約}

周期境界条件を考慮すると, 粒子-粒子間相互ポテンシャルの総計はまず以下のように書ける.

\begin{align}
  \sum_{n_x \in \mathbb{Z}} \sum_{i=1}^{N} \sum_{\substack{j=1 \\ (j \neq i \ \text{for} \ n_{x} = 0)}}^{N} \frac{1}{2} \phi_{\text{LJ}}^{\text{pair}}(|\bm{r}_i -(\bm{r}_j + L_x \bm{e}_x)|)
\end{align}

ここで, $n_x = 0$; オリジナルセルの中では, 同じ$i,\ j$ペアのポテンシャルエネルギーを2回足すことになるので, ポテンシャルを$1/2$している. その上で, $j = i$の場合は自分自身との相互作用になるため, これは除外する. $n_x \neq 0$の場合, 粒子$j$はイメージ粒子となるため, $j=i$の場合も含めることになる. このときにもダブルカウントがあるので, ポテンシャルを$1/2$している.

\begin{figure}[H]
  \centering
  \caption{}
  \label{fig:system_periodic}
  \includegraphics[scale=0.7]{image/system_periodic.jpg}
\end{figure}

注目する系の粒子が常にオリジナルセルの中にとどまっているかのようにMD上で扱うには,

\begin{align}
  x_{i} = x_{i}' \mod L_{x}
\end{align}

のように,  飛び出した粒子の$x$座標$x_{i}'$を上式のように$x_i$にシフトすれば良い. しかし, 周期境界条件とセットに, 最近接イメージ規約として, 粒子$i$がオリジナル粒子と各イメージ粒子の中で最も近い粒子$j$らとのみ相互作用をすることを課すと, 粒子間の相互ポテンシャルの総計は先ほどよりも簡単に書けるようになる.

\begin{align}
  \sum_{i=1}^{N} \sum_{j > i}^{N} \phi_{\text{LJ}}^{\text{pair}} (r_{ij})
\end{align}

\subsubsection{壁-粒子間相互作用ポテンシャル}

\begin{align}
  \phi_{\text{LJ}}^{\text{wall}}(r; \varepsilon^{\text{wall}}, \sigma^{\text{wall}}) = 4\varepsilon^{\text{wall}} \qty[\qty(\frac{\sigma^{\text{wall}}}{r})^{12} - \qty(\frac{\sigma^{\text{wall}}}{r})^{6} ] \\
\end{align}

パラメータは

\begin{align}
  \varepsilon^{\text{wall}} &= \qty(1.0 - \text{R}_\text{d}) \times \varepsilon \\
  \sigma^{\text{wall}} &= \qty(0.5 + \text{R}_\text{t}) \times \sigma \\
  r^{\text{wall}}_{\text{cut}} &= \qty(2^{1/6} + \text{R}_\text{a}) \times \sigma^{\text{wall}}
\end{align}

カットオフ長とシフトアップを考慮して

\begin{align}
  \tilde{\phi}_{\text{LJ}}^{\text{wall}}(r;r_{\text{cut}}^{\text{wall}}) = \qty{\phi_{\text{LJ}}^{\text{wall}}(r) - \phi_{\text{LJ}}^{\text{wall}}(r_{\text{cut}}^{\text{wall}})}\theta \qty(r_{\text{cut}}^{\text{wall}}-r) \\
\end{align}

この系では, $y$軸方向に壁がついている. よって, 壁ポテンシャルは

\begin{align}
  V^{\text{wall}}(y; L_y) &= \tilde{\phi}_{\text{LJ}}^{\text{wall}}(y;r_{\text{cut}}^{\text{wall}}) + \tilde{\phi}_{\text{LJ}}^{\text{wall}}(L_y - y;r_{\text{cut}}^{\text{wall}})
\end{align}

のように書ける. これまでのことより, ハミルトニアンは以下のように書き表せる.

\begin{align}
  \label{Hamiltonian}
    H(\Gamma; g)
    &= \sum_{i=1}^{N}
    \left[
      \frac{{\bm{p}_i}^2}{2m} 
      + \sum_{j > i}^{N}
        \tilde{\phi}_{\text{LJ}}^{\text{pair}}(r_{ij})
      + mgy_i +V^{\text{wall}}(y_i)
    \right]
\end{align}

\subsection{熱流}

langevin熱浴に侵入した粒子に対しては, brownian 動力学計算を実行する(\url{https://docs.lammps.org/fix_langevin.html}). その粒子$i$にはたらく力$\bm{F}_i$はLAMMPSのドキュメントに則った形式だと以下のように書き表せる.

\begin{align}
  \bm{F}_i &= \bm{F}_i^c + \bm{F}_i^f + \bm{F}_i^r \\
  \bm{F}_i^c &= - \bm{\nabla} 
  \left[
    \sum_{j > i}^{N}
        \tilde{\phi}_{\text{LJ}}^{\text{pair}}(r_{ij})
      + mgy_i +V^{\text{wall}}(y_i)
  \right]  \\
  \bm{F}_i^f &= -\frac{m_i}{\text{damp}}\bm{v}_i \\
  F_i^r &\propto \sqrt{\frac{k_\text{B} Tm_i}{dt \text{damp}}}
\end{align}

それぞれの力の説明を記す.

\begin{itemize}
  \item $\bm{F}^c$; ポテンシャルを介して計算される力
  \item $\bm{F}^f$; 摩擦力
  \item $\bm{F}^r$; ランダム力
\end{itemize}


\subsubsection{温度制御}


2d kinetic temperature

\begin{align}
  T \equiv \frac{1}{Nk_{\text{B}}}\sum_{i=1}^{N} \frac{1}{2}m_i v_{i}^2
\end{align}

粒子$i$が熱浴に侵入すると, その粒子の運動はランジュバン方程式に従う. 侵入していないときは, $\gamma = 0$になり, 正準方程式に等しくなる.

\begin{align}
  \dot{\bm{r}_i} &= \pdv{H}{\bm{p}_i} \\
  \dot{\bm{p}_i} &= - \pdv{H}{\bm{r}_i} - \gamma \dot{\bm{r}_i} + \sqrt{2 \gamma k_{\text{B}}T_{\nu}}\bm{\xi}_i (t) \\
  \expval{\xi_{i}^{a}(t)} &= 0 \\
  \expval{\xi_{i}^{a}(t)\xi_{j}^{b}(t')} &= \delta_{i,j} \delta_{a,b}\delta (t-t') 
\end{align}

\begin{align}
  \gamma(y_i) &= 1. \ T_{\nu}(y_i) = T_{\text{H}}. \ (0 < y_i < 8\sigma) \\
  \gamma(y_i) &= 1. \ T_{\nu}(y_i) = T_{\text{C}}. \ (L_y - 8\sigma < y_i < L_y) \\
  \gamma(y_i) &= 0. \ T_{\nu}(y_i) = T. \ (8\sigma < y_i < L_y - 8\sigma)
\end{align}

\section{実験}

雨が降るシミュレーションをしたい.

系の両端のポテンシャルエネルギー差$mgL_y$と運動エネルギー差$k_{\text{B}}\Delta T$の比を$\chi$として以下のように設定する.

\begin{align}
  \chi \equiv \frac{k_{\text{B}}\Delta T}{mgL_{y}} = 1.265
\end{align}

壁ポテンシャルまわりの無次元パラメータを3つ用意する.

\begin{align}
  \text{R}_\text{d} &: 乾き具合. \\
  \text{R}_\text{t} &: 壁の厚み. \\
  \text{R}_\text{a} &: 濡れ具合.
\end{align}

これを用いて, 壁-粒子間相互作用LJポテンシャルは以下のように書き表す.

\begin{align}
  \varepsilon^{\text{wall}} &= \qty(1.0 - \text{R}_\text{d}) \times \varepsilon \\
  \sigma^{\text{wall}} &= \qty(0.5 + \text{R}_\text{t}) \times \sigma \\
  r^{\text{wall}}_{\text{cut}} &= \qty(2^{1/6} + \text{R}_\text{a}) \times \sigma^{\text{wall}}
\end{align}

パラメータ $(\text{R}_\text{d}, \text{R}_\text{t}, \text{R}_\text{a})$ を変えることによって, 壁-粒子間相互作用LJポテンシャルが変わったときに, どのように粒子集団の様相が変化するかをみる. 本章の以降の実験は特記がない限り以下のパラメータに近い値で行うものとする. 

\begin{itemize}
  \item $N = 1250$
  \item $\rho {\sigma}^2 = 0.4$
  \item $L_x / \sigma = 39.5\dots$
  \item $L_y / \sigma = 79.0\dots$
  \item $k_{\text{B}} T / \varepsilon = 0.43$
  \item $k_{\text{B}} \Delta T / \varepsilon = 0.04$
  \item $mg\sigma/\varepsilon = 4.0 \times 10^{-4}$
  \item $t_f \sqrt{\varepsilon / m \sigma^2} = 1.0 \times 10^{5}$
\end{itemize}


\subsection{追実験}

壁を完全に濡らしていることを考えたいので, $\sigma^{\text{wall}}=1.0\neq 0.5$ のときを考えている.

重力をかけた状態で緩和するまでシミュレーションを行っている.

\begin{itemize}
  \item $N = 5000$
  \item $\rho \sigma^2 = 0.4$
  \item $L_x / \sigma = 79.0\dots$
  \item $L_y / \sigma = 158.1\dots$
  \item $k_{\text{B}} T/\varepsilon = 4.3$
  \item $k_{\text{B}} \Delta T/\varepsilon = 0.0$
  \item $mg\sigma/\varepsilon = 2.0 \times 10^{-4}$
  \item $t_b \sqrt{\varepsilon / m \sigma^2} = 5.0 \times 10^{5}$
\end{itemize}

重力をかけた緩和後の系で, 熱浴の温度差を改めて以下のようにつけ, 熱流を流してシミュレーションをしている.

\begin{itemize}
  \item $\chi = k_{\text{B}}\Delta T / mg L_y = 1.265$
  \item $k_{\text{B}} \Delta T/\varepsilon = 0.04$
  \item $t_a \sqrt{\varepsilon / m \sigma^2} = 5.0 \times 10^{5}$
\end{itemize}


まずは, $(\text{R}_\text{d} = 0.0, \text{R}_\text{t} = 0.5, \text{R}_\text{a} = 3.0 - 2^{1/6})$ で実験を行う. これは, $(\varepsilon^{\text{wall}} = \varepsilon, \sigma^{\text{wall}} = \sigma, r^{\text{wall}}_{\text{cut}} = 3\sigma^{\text{wall}})$ に対応しているので, $\phi_{\text{LJ}} = \phi_{\text{LJ}}^{\text{wall}}$ ということになり, 先行研究と同じ結果を得るはずである.

\begin{figure}[H]
  \centering
  \href{https://youtu.be/fxn1mU1ZZFQ}{\includegraphics[scale=0.2]{image/2023-11-13T17:41:52.785__chi1.265_Ay50_rho0.4_T0.43_dT0.04_Rd0.0_Rt0.5_Ra1.877538_g0.0003999718779659611_run2.0e7_output.png}}
  \caption{Ay50\_rho0.4\_T0.43\_dT0.04\_Rd0.0\_Rt0.5\_Ra1.877538\_g0.0004\_run2.0e7}
  \label{}
\end{figure}


\subsection{$\text{R}_\text{a}$, $\text{R}_\text{t}$ マップ}


\begin{align}
  \text{R}_\text{a} &= 0.0 \sim 3.0 - 2^{1/6} \\
  \text{R}_\text{t} &= 0.0 \sim 0.5
\end{align}

\begin{figure}[H]
  \centering
  \begin{tabular}{ccc}
    \begin{minipage}[t]{0.2\hsize}
      \centering
      \includegraphics[width=\textwidth]{image/RaRtmap_LJ/LJ-Potential_Rt0.0.png}
      \subcaption{$\text{R}_\text{t}:0.0$}
      \label{}
    \end{minipage} &
    \begin{minipage}[t]{0.2\hsize}
      \centering
      \includegraphics[width=\textwidth]{image/RaRtmap_LJ/LJ-Potential_Rt0.125.png}
      \subcaption{$\text{R}_\text{t}:0.125$}
      \label{}
    \end{minipage} &
    \begin{minipage}[t]{0.2\hsize}
      \centering
      \includegraphics[width=\textwidth]{image/RaRtmap_LJ/LJ-Potential_Rt0.25.png}
      \subcaption{$\text{R}_\text{t}:0.25$}
      \label{}
    \end{minipage} \\
    \begin{minipage}[t]{0.2\hsize}
      \centering
      \includegraphics[width=\textwidth]{image/RaRtmap_LJ/LJ-Potential_Rt0.375.png}
      \subcaption{$\text{R}_\text{t}:0.375$}
      \label{}
    \end{minipage} &
    \begin{minipage}[t]{0.2\hsize}
      \centering
      \includegraphics[width=\textwidth]{image/RaRtmap_LJ/LJ-Potential_Rt0.5.png}
      \subcaption{$\text{R}_\text{t}:0.5$}
      \label{}
    \end{minipage} 
  \end{tabular}
  \caption{LJ-potential}
  \label{}
\end{figure}

\begin{itemize}
  \item $N = 1250$
  \item $\rho {\sigma}^2 = 0.4$
  \item $L_x / \sigma = 39.5\dots$
  \item $L_y / \sigma = 79.0\dots$
  \item $k_{\text{B}} T / \varepsilon = 0.43$
  \item $k_{\text{B}} \Delta T / \varepsilon = 0.04$
  \item $mg\sigma/\varepsilon = 4.0 \times 10^{-4}$
  \item $t_f \sqrt{\varepsilon / m \sigma^2} = 2.0 \times 10^{5}$
\end{itemize}

\input{subtex/RaRtmap_movie}

重心位置をスケーリングして, 時系列プロットすると,

\input{subtex/RaRtmap_time.tex}

重心位置の揺らぎについて同時プロットで表してみる.

\begin{figure}[H]
  \centering
  \includegraphics[scale=0.75]{image/lnStdYg_Ra.png}
  \caption{}
  \label{}
\end{figure}

\appendix
\section{\href{https://github.com/m-agnet/Report.git}{ソースコード}}
% ここにソースコードを追加.
\lstinputlisting[caption=in.vdw\_mod]{src/in.vdw2d_mod}
\lstinputlisting[caption=lammps\_modexe.jl]{src/lammps_modexe.jl}
\lstinputlisting[caption=plot\_LJpotential.jl]{src/plot_LJpotential.jl}


% 参考文献
\bibliographystyle{jplain}
\bibliography{seminar}



\end{document}